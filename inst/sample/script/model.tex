\input{settings.sty}
\usepackage{Sweave}

 
\begin{document}
\vspace*{2cm}
\begin{center}
{\Large MIfuns Sample Script}\\
\vspace{1.5cm}
{\Large Phase I Modeling}\\
~\\
\today\\
~\\
Tim Bergsma\\
\end{center}
\newpage

\section{Purpose}
This script runs NONMEM models and diagnostics for sample phase1 data.
\section{Model Development}
\subsection{Set up for NONMEM run.}
\begin{Schunk}
\begin{Sinput}
> getwd()
\end{Sinput}
\begin{Soutput}
[1] "/Users/timb/project/metrum-mifuns/inst/sample/script"
\end{Soutput}
\begin{Sinput}
> library(MIfuns)
\end{Sinput}
\begin{Soutput}
MIfuns 4.1.0 
\end{Soutput}
\begin{Sinput}
> command <- '/common/NONMEM/nm7_osx1/test/nm7_osx1.pl'
> cat.cov='SEX'
> cont.cov=c('HEIGHT','WEIGHT','AGE')
> par.list=c('CL','Q','KA','V','V2','V3')
> eta.list=paste('ETA',1:10,sep='')
\end{Sinput}
\end{Schunk}
\subsection{Run NONMEM.}
To force a re-run of this model, delete 1005/diagnostics.pdf.
\begin{Schunk}
\begin{Sinput}
> if(!file.exists('../nonmem/1005/diagnostics.pdf'))NONR(
+      run=1005,
+      command=command,
+      project='../nonmem',
+      grid=TRUE,
+      nice=TRUE,
+      checkrunno=FALSE,
+      cont.cov=cont.cov,
+      cat.cov=cat.cov,
+      par.list=par.list,
+      eta.list=eta.list,
+      plotfile='../nonmem/*/diagnostics.pdf',
+      streams='../nonmem/ctl'
+ )
> getwd()
\end{Sinput}
\begin{Soutput}
[1] "/Users/timb/project/metrum-mifuns/inst/sample/script"
\end{Soutput}
\begin{Sinput}
> while(!file.exists('../nonmem/1005/diagnostics.pdf')){}
\end{Sinput}
\end{Schunk}
Covariance succeeded on model 1005.
\section{Predictive Check}
\subsection{Create a simulation control stream.}
Convert control stream to R object.
\begin{Schunk}
\begin{Sinput}
> ctl <- read.nmcontrol('../nonmem/ctl/1005.ctl')
\end{Sinput}
\end{Schunk}
Strip comments and view.
\begin{Schunk}
\begin{Sinput}
> ctl[] <- lapply(ctl,function(rec)sub(' *;.*','',rec))
> ctl
\end{Sinput}
\begin{Soutput}
 [1] "$PROB 1005 phase1 2 CMT like 1004 but diff. initial on V3"                                   
 [2] "$INPUT C ID TIME SEQ=DROP EVID AMT DV SUBJ HOUR TAFD TAD LDOS MDV HEIGHT WT SEX AGE DOSE FED"
 [3] "$DATA ../../data/derived/phase1.csv IGNORE=C"                                                
 [4] "$SUBROUTINE ADVAN4 TRANS4"                                                                   
 [5] "$PK  CL=THETA(1)*EXP(ETA(1)) * THETA(6)**SEX * (WT/70)**THETA(7)"                            
 [6] " V2 =THETA(2)*EXP(ETA(2))"                                                                   
 [7] " KA=THETA(3)*EXP(ETA(3))"                                                                    
 [8] " Q  =THETA(4)"                                                                               
 [9] " V3=THETA(5)"                                                                                
[10] " S2=V2"                                                                                      
[11] " "                                                                                           
[12] "$ERROR  Y=F*EXP(ERR(1))"                                                                     
[13] " IPRE=F"                                                                                     
[14] ""                                                                                            
[15] "$THETA (0,10,50)"                                                                            
[16] "(0,10,100)"                                                                                  
[17] "(0,0.2, 5)"                                                                                  
[18] "(0,10,50)"                                                                                   
[19] "(0,100,1000)"                                                                                
[20] "(0,1,2)"                                                                                     
[21] "(0,0.75,3)"                                                                                  
[22] ""                                                                                            
[23] "$OMEGA 0.09 0.09 0.09 "                                                                      
[24] ""                                                                                            
[25] ""                                                                                            
[26] ""                                                                                            
[27] ""                                                                                            
[28] ""                                                                                            
[29] "$SIGMA 0.09"                                                                                 
[30] ""                                                                                            
[31] ""                                                                                            
[32] ""                                                                                            
[33] "$ESTIMATION MAXEVAL=9999 PRINT=5 NOABORT METHOD=1 INTER MSFO=./1005.msf"                     
[34] "$COV PRINT=E"                                                                                
[35] "$TABLE NOPRINT FILE=./1005.tab ONEHEADER ID AMT TIME EVID PRED IPRE CWRES"                   
[36] "$TABLE NOPRINT FILE=./1005par.tab ONEHEADER ID TIME CL Q V2 V3 KA ETA1 ETA2 ETA3"            
\end{Soutput}
\end{Schunk}
Fix records of interest.
\begin{Schunk}
\begin{Sinput}
> ctl$prob
\end{Sinput}
\begin{Soutput}
[1] "1005 phase1 2 CMT like 1004 but diff. initial on V3"
\end{Soutput}
\begin{Sinput}
> ctl$prob <- sub('1005','1105',ctl$prob)
> names(ctl)
\end{Sinput}
\begin{Soutput}
 [1] "prob"       "input"      "data"       "subroutine" "pk"        
 [6] "error"      "theta"      "omega"      "sigma"      "estimation"
[11] "cov"        "table"      "table"     
\end{Soutput}
\begin{Sinput}
> names(ctl)[names(ctl)=='theta'] <- 'msfi'
> ctl$msfi <- '=../1005/1005/msf'
> ctl$omega <- NULL
> ctl$sigma <- NULL
> names(ctl)[names(ctl)=='estimation'] <- 'simulation'
> ctl$simulation <- 'ONLYSIM (1968) SUBPROBLEMS=500'
> ctl$cov <- NULL
> ctl$table <- NULL
> ctl$table <- NULL
> ctl$table <- 'DV NOHEADER NOPRINT FILE=./1105.tab FORWARD NOAPPEND'
> write.nmcontrol('../nonmem/ctl/1105.ctl')
\end{Sinput}
\end{Schunk}
\subsection{Run the simulation.}
This run makes the predictions (simulations).
\begin{Schunk}
\begin{Sinput}
> if(!file.exists('../nonmem/1105/1105.lst'))NONR(
+      run=1105,
+      command=command,
+      project='../nonmem',
+      grid=TRUE,
+      nice=TRUE,
+      diag=FALSE,
+      streams='../nonmem/ctl'
+ )
> getwd()
\end{Sinput}
\begin{Soutput}
[1] "/Users/timb/project/metrum-mifuns/inst/sample/script"
\end{Soutput}
\begin{Sinput}
> while(!file.exists('../nonmem/1105/1105.lst')){}
\end{Sinput}
\end{Schunk}
\subsection{Recover and format the original dataset.}
Now we fetch the results and integrate them with the other data.
\begin{Schunk}
\begin{Sinput}
> phase1 <- read.csv('../data/derived/phase1.csv',na.strings='.')
> head(phase1)
\end{Sinput}
\begin{Soutput}
     C ID TIME SEQ EVID  AMT    DV SUBJ HOUR TAFD  TAD LDOS MDV HEIGHT WEIGHT
1    C  1 0.00   0    0   NA 0.000    1 0.00 0.00   NA   NA   0    174   74.2
2 <NA>  1 0.00   1    1 1000    NA    1 0.00 0.00 0.00 1000   1    174   74.2
3 <NA>  1 0.25   0    0   NA 0.363    1 0.25 0.25 0.25 1000   0    174   74.2
4 <NA>  1 0.50   0    0   NA 0.914    1 0.50 0.50 0.50 1000   0    174   74.2
5 <NA>  1 1.00   0    0   NA 1.120    1 1.00 1.00 1.00 1000   0    174   74.2
6 <NA>  1 2.00   0    0   NA 2.280    1 2.00 2.00 2.00 1000   0    174   74.2
  SEX  AGE DOSE FED SMK DS CRCN predose zerodv
1   0 29.1 1000   1   0  0 83.5       1      1
2   0 29.1 1000   1   0  0 83.5       0      0
3   0 29.1 1000   1   0  0 83.5       0      0
4   0 29.1 1000   1   0  0 83.5       0      0
5   0 29.1 1000   1   0  0 83.5       0      0
6   0 29.1 1000   1   0  0 83.5       0      0
\end{Soutput}
\begin{Sinput}
> phase1 <- phase1[is.na(phase1$C),c('SUBJ','TIME','DV')]
> records <- nrow(phase1)
> records
\end{Sinput}
\begin{Soutput}
[1] 550
\end{Soutput}
\begin{Sinput}
> phase1 <- phase1[rep(1:records,500),]
> nrow(phase1)
\end{Sinput}
\begin{Soutput}
[1] 275000
\end{Soutput}
\begin{Sinput}
> phase1$SIM <- rep(1:500,each=records)
> #head(phase1,300)
> with(phase1,DV[SIM==1 & SUBJ==12])
\end{Sinput}
\begin{Soutput}
 [1]     NA  2.260  2.830  8.730 19.300 15.200 16.200  8.830 12.900 12.700
[11]  7.140  5.740  1.980  0.791
\end{Soutput}
\begin{Sinput}
> with(phase1,DV[SIM==2 & SUBJ==12])
\end{Sinput}
\begin{Soutput}
 [1]     NA  2.260  2.830  8.730 19.300 15.200 16.200  8.830 12.900 12.700
[11]  7.140  5.740  1.980  0.791
\end{Soutput}
\end{Schunk}
\subsection{Recover and format the simulation results.}
\begin{Schunk}
\begin{Sinput}
> pred <- scan('../nonmem/1105/1105.tab')
> nrow(phase1)
\end{Sinput}
\begin{Soutput}
[1] 275000
\end{Soutput}
\begin{Sinput}
> length(pred)
\end{Sinput}
\begin{Soutput}
[1] 275000
\end{Soutput}
\end{Schunk}
\subsection{Combine the original data and the simulation data.}
\begin{Schunk}
\begin{Sinput}
> phase1$PRED <- pred
> head(phase1)
\end{Sinput}
\begin{Soutput}
  SUBJ TIME    DV SIM    PRED
2    1 0.00    NA   1 0.00000
3    1 0.25 0.363   1 0.17932
4    1 0.50 0.914   1 0.53642
5    1 1.00 1.120   1 0.78983
6    1 2.00 2.280   1 1.84990
7    1 3.00 1.630   1 1.96530
\end{Soutput}
\begin{Sinput}
> phase1 <- phase1[!is.na(phase1$DV),]
> head(phase1)
\end{Sinput}
\begin{Soutput}
  SUBJ TIME    DV SIM    PRED
3    1 0.25 0.363   1 0.17932
4    1 0.50 0.914   1 0.53642
5    1 1.00 1.120   1 0.78983
6    1 2.00 2.280   1 1.84990
7    1 3.00 1.630   1 1.96530
8    1 4.00 2.040   1 2.01810
\end{Soutput}
\end{Schunk}
\subsection{Plot predictive checks.}
\subsubsection{Aggregate data within subject.}
Since subjects may contribute differing numbers of observations, it may
be useful to look at predictions from a subject-centric perspective.
Therefore, we wish to calculate summary statistics for each subject, 
(observed and predicted) and then make obspred comparisons therewith.
\begin{Schunk}
\begin{Sinput}
> head(phase1)
\end{Sinput}
\begin{Soutput}
  SUBJ TIME    DV SIM    PRED
3    1 0.25 0.363   1 0.17932
4    1 0.50 0.914   1 0.53642
5    1 1.00 1.120   1 0.78983
6    1 2.00 2.280   1 1.84990
7    1 3.00 1.630   1 1.96530
8    1 4.00 2.040   1 2.01810
\end{Soutput}
\begin{Sinput}
> subject <- melt(phase1,measure.var=c('DV','PRED'))
> head(subject)
\end{Sinput}
\begin{Soutput}
  SUBJ TIME SIM variable value
1    1 0.25   1       DV 0.363
2    1 0.50   1       DV 0.914
3    1 1.00   1       DV 1.120
4    1 2.00   1       DV 2.280
5    1 3.00   1       DV 1.630
6    1 4.00   1       DV 2.040
\end{Soutput}
\end{Schunk}
We are going to aggregate each subject's DV and PRED values using cast().
cast() likes an aggregation function that returns a list.
We write one that grabs min med max for each subject, sim, and variable.
\begin{Schunk}
\begin{Sinput}
> metrics <- function(x)list(min=min(x), med=median(x), max=max(x))
\end{Sinput}
\end{Schunk}
Now we cast, ignoring time.
\begin{Schunk}
\begin{Sinput}
> subject <- data.frame(cast(subject, SUBJ + SIM + variable ~ .,fun=metrics))
> head(subject)
\end{Sinput}
\begin{Soutput}
  SUBJ SIM variable      min    med    max
1    1   1       DV 0.363000 1.6100 3.0900
2    1   1     PRED 0.179320 1.9653 5.0314
3    1   2       DV 0.363000 1.6100 3.0900
4    1   2     PRED 0.096462 3.0448 7.4728
5    1   3       DV 0.363000 1.6100 3.0900
6    1   3     PRED 0.450430 5.5284 8.7665
\end{Soutput}
\end{Schunk}
Note that regardless of SIM, DV (observed) is constant.

Now we melt the metrics.
\begin{Schunk}
\begin{Sinput}
> metr <- melt(subject,measure.var=c('min','med','max'),variable_name='metric')
> head(metr)
\end{Sinput}
\begin{Soutput}
  SUBJ SIM variable metric    value
1    1   1       DV    min 0.363000
2    1   1     PRED    min 0.179320
3    1   2       DV    min 0.363000
4    1   2     PRED    min 0.096462
5    1   3       DV    min 0.363000
6    1   3     PRED    min 0.450430
\end{Soutput}
\begin{Sinput}
> metr$value <- reapply(
+ 	metr$value,
+ 	INDEX=metr[,c('SIM','variable','metric')],
+ 	FUN=sort,
+ 	na.last=FALSE
+ )
> metr <- data.frame(cast(metr))
> head(metr)
\end{Sinput}
\begin{Soutput}
  SUBJ SIM metric    DV     PRED
1    1   1    min 0.139 0.064213
2    1   1    med 1.025 1.943600
3    1   1    max 2.530 3.945400
4    1   2    min 0.139 0.016162
5    1   2    med 1.025 1.476300
6    1   2    max 2.530 3.463200
\end{Soutput}
\begin{Sinput}
> nrow(metr)
\end{Sinput}
\begin{Soutput}
[1] 60000
\end{Soutput}
\begin{Sinput}
> metr <- metr[!is.na(metr$DV),]#maybe no NA
> nrow(metr)
\end{Sinput}
\begin{Soutput}
[1] 60000
\end{Soutput}
\end{Schunk}
We plot using lattice.
\begin{Schunk}
\begin{Sinput}
> print(
+ 	xyplot(
+ 		PRED~DV|metric,
+ 		metr,
+ 		groups=SIM,
+ 		scales=list(relation='free'),
+ 		type='l',
+ 		panel=function(...){
+ 			panel.superpose(...)
+ 			panel.abline(0,1,col='white',lwd=2)
+ 		}
+ 	)
+ )
\end{Sinput}
\end{Schunk}
\includegraphics{model-qq}

For detail, we show one endpoint, tossing the outer 5 percent of values, and 
indicating quartiles.
\begin{Schunk}
\begin{Sinput}
> med <- metr[metr$metric=='med',]
> med$metric <- NULL
> head(med)
\end{Sinput}
\begin{Soutput}
   SUBJ SIM    DV     PRED
2     1   1 1.025 1.943600
5     1   2 1.025 1.476300
8     1   3 1.025 1.466300
11    1   4 1.025 1.342400
14    1   5 1.025 1.362350
17    1   6 1.025 0.625815
\end{Soutput}
\begin{Sinput}
> trim <- inner(med, id.var=c('SIM'),measure.var=c('PRED','DV'))
> head(trim)
\end{Sinput}
\begin{Soutput}
  SIM DV PRED
1   1 NA   NA
2   2 NA   NA
3   3 NA   NA
4   4 NA   NA
5   5 NA   NA
6   6 NA   NA
\end{Soutput}
\begin{Sinput}
> nrow(trim)
\end{Sinput}
\begin{Soutput}
[1] 20000
\end{Soutput}
\begin{Sinput}
> trim <- trim[!is.na(trim$DV),]
> nrow(trim)
\end{Sinput}
\begin{Soutput}
[1] 19000
\end{Soutput}
\begin{Sinput}
> head(trim)
\end{Sinput}
\begin{Soutput}
    SIM   DV   PRED
501   1 1.13 1.9653
502   2 1.13 1.5989
503   3 1.13 1.4754
504   4 1.13 1.4074
505   5 1.13 1.3787
506   6 1.13 1.4753
\end{Soutput}
\begin{Sinput}
> print(
+ 	xyplot(
+ 		PRED~DV,
+ 		trim,
+ 		groups=SIM,
+ 		type='l',
+ 		panel=function(x,y,...){
+ 			panel.xyplot(x=x,y=y,...)
+ 			panel.abline(0,1,col='white',lwd=2)
+ 			panel.abline(
+ 				v=quantile(x,probs=c(0.25,0.5,0.75)),
+ 				col='grey',
+ 				lty=2
+ 			)
+ 		}
+ 	)
+ )
\end{Sinput}
\end{Schunk}
\includegraphics{model-qqdetail}

We also show densityplots of predictions at those quartiles.
\begin{Schunk}
\begin{Sinput}
> head(trim)
\end{Sinput}
\begin{Soutput}
    SIM   DV   PRED
501   1 1.13 1.9653
502   2 1.13 1.5989
503   3 1.13 1.4754
504   4 1.13 1.4074
505   5 1.13 1.3787
506   6 1.13 1.4753
\end{Soutput}
\begin{Sinput}
> quantile(trim$DV)
\end{Sinput}
\begin{Soutput}
    0%    25%    50%    75%   100% 
  1.13   7.69  20.25 104.00 332.00 
\end{Soutput}
\begin{Sinput}
> molt <- melt(trim, id.var='SIM')
> head(molt)
\end{Sinput}
\begin{Soutput}
  SIM variable value
1   1       DV  1.13
2   2       DV  1.13
3   3       DV  1.13
4   4       DV  1.13
5   5       DV  1.13
6   6       DV  1.13
\end{Soutput}
\begin{Sinput}
> quart <- data.frame(cast(molt,SIM+variable~.,fun=quantile,probs=c(0.25,0.5,0.75)))
> head(quart)
\end{Sinput}
\begin{Soutput}
  SIM variable      X25.    X50.      X75.
1   1       DV  7.950000 20.2500 100.10000
2   1     PRED 10.329750 22.8675  91.61825
3   2       DV  7.950000 20.2500 100.10000
4   2     PRED 10.241500 23.4225  97.26175
5   3       DV  7.950000 20.2500 100.10000
6   3     PRED  8.081437 20.0330 106.59750
\end{Soutput}
\begin{Sinput}
> molt <- melt(quart,id.var='variable',measure.var=c('X25.','X50.','X75.'),variable_name='quartile')
> head(molt)
\end{Sinput}
\begin{Soutput}
  variable quartile     value
1       DV     X25.  7.950000
2     PRED     X25. 10.329750
3       DV     X25.  7.950000
4     PRED     X25. 10.241500
5       DV     X25.  7.950000
6     PRED     X25.  8.081437
\end{Soutput}
\begin{Sinput}
> levels(molt$quartile)
\end{Sinput}
\begin{Soutput}
[1] "X25." "X50." "X75."
\end{Soutput}
\begin{Sinput}
> levels(molt$quartile) <- c('first quartile','second quartile','third quartile')
> head(molt)
\end{Sinput}
\begin{Soutput}
  variable       quartile     value
1       DV first quartile  7.950000
2     PRED first quartile 10.329750
3       DV first quartile  7.950000
4     PRED first quartile 10.241500
5       DV first quartile  7.950000
6     PRED first quartile  8.081437
\end{Soutput}
\begin{Sinput}
> levels(molt$variable)
\end{Sinput}
\begin{Soutput}
[1] "DV"   "PRED"
\end{Soutput}
\begin{Sinput}
> molt$variable <- factor(molt$variable,levels=c('PRED','DV'))
> print(
+ 	densityplot(
+ 		~value|quartile,
+ 		molt,
+ 		groups=variable,
+ 		layout=c(3,1),
+ 		scales=list(relation='free'),
+ 		aspect=1,
+ 		panel=panel.superpose,
+ 		panel.groups=function(x,...,group.number){
+ 			if(group.number==1)panel.densityplot(x,...)
+ 			if(group.number==2)panel.abline(v=unique(x),...)
+ 		},
+ 		auto.key=TRUE
+ 	)
+ )
\end{Sinput}
\end{Schunk}
\includegraphics{model-qqdensity}
\section{Bootstrap Estimates of Parameter Uncertainty}
\subsection{Create directories.}
\begin{Schunk}
\begin{Sinput}
> getwd()
\end{Sinput}
\begin{Soutput}
[1] "/Users/timb/project/metrum-mifuns/inst/sample/script"
\end{Soutput}
\begin{Sinput}
> dir.create('../nonmem/1005.boot')
> dir.create('../nonmem/1005.boot/data')
> dir.create('../nonmem/1005.boot/ctl')
\end{Sinput}
\end{Schunk}
\subsection{Create replicate control streams.}
\begin{Schunk}
\begin{Sinput}
> t <- metaSub(
+      clear(readLines('../nonmem/ctl/1005.ctl'),';.+',fixed=FALSE),
+      names=1:300,
+      pattern=c(
+          '1005',
+          '../../data/derived/phase1.csv',
+          '$COV',
+          '$TABLE'
+      ),
+      replacement=c(
+          '*',
+          '../data/*.csv',
+          ';$COV',
+          ';$TABLE'
+     ),
+     fixed=TRUE,
+     out='../nonmem/1005.boot/ctl',
+     suffix='.ctl'
+  )
\end{Sinput}
\end{Schunk}
\subsection{Create replicate data sets by resampling original.}
\begin{Schunk}
\begin{Sinput}
>  bootset <- read.csv('../data/derived/phase1.csv')
>  r <- resample(
+  	bootset,
+  	names=1:300,
+  	key='ID',
+  	rekey=TRUE,
+  	out='../nonmem/1005.boot/data',
+  	stratify='SEX'
+  )
\end{Sinput}
\end{Schunk}
\subsection{Run bootstrap models.}
To force a re-run of bootstraps, delete log.csv.
\begin{Schunk}
\begin{Sinput}
> if(!file.exists('../nonmem/1005.boot/CombRunLog.csv'))NONR(
+      run=1:300,
+      command=command,
+      project='../nonmem/1005.boot/',
+      boot=TRUE,
+      nice=TRUE,
+      streams='../nonmem/1005.boot/ctl'
+ )
> getwd()  
\end{Sinput}
\begin{Soutput}
[1] "/Users/timb/project/metrum-mifuns/inst/sample/script"
\end{Soutput}
\end{Schunk}
\subsection{Summarize bootstrap models.}
When the bootstraps are complete, we return here and summarize. If you 
do not have time for bootstraps, use read.csv() on ../nonmem/1005.boot/log.csv.
\begin{Schunk}
\begin{Sinput}
> #wait for bootstraps to finish
> #while(!(all(file.exists(paste(sep='','../nonmem/1005.boot/',1:300,'.boot/',1:300,'.lst'))))){}
> if(file.exists('../nonmem/1005.boot/log.csv')){
+     boot <- read.csv('../nonmem/1005.boot/log.csv',as.is=TRUE)
+ }else{
+     boot <- rlog(
+ 	run=1:300,
+ 	project='../nonmem/1005.boot',
+ 	boot=TRUE,
+ 	append=FALSE,
+ 	tool='nm7'
+     )
+     write.csv(boot, '../nonmem/1005.boot/log.csv')
+ }
> head(boot)
\end{Sinput}
\begin{Soutput}
  X tool run parameter   moment
1 1  nm7   1      prob     text
2 2  nm7   1       min   status
3 3  nm7   1       ofv  minimum
4 4  nm7   1    THETA1 estimate
5 5  nm7   1    THETA1     prse
6 6  nm7   1    THETA2 estimate
                                             value
1 1 phase1 2 CMT like 1004 but diff. initial on V3
2                                                0
3                                 2760.84241850239
4                                          7.98893
5                                             <NA>
6                                           19.892
\end{Soutput}
\begin{Sinput}
> unique(boot$parameter)
\end{Sinput}
\begin{Soutput}
 [1] "prob"     "min"      "ofv"      "THETA1"   "THETA2"   "THETA3"  
 [7] "THETA4"   "THETA5"   "THETA6"   "THETA7"   "OMEGA1.1" "OMEGA2.1"
[13] "OMEGA2.2" "OMEGA3.1" "OMEGA3.2" "OMEGA3.3" "SIGMA1.1"
\end{Soutput}
\begin{Sinput}
> text2decimal(unique(boot$parameter))
\end{Sinput}
\begin{Soutput}
 [1]  NA  NA  NA 1.0 2.0 3.0 4.0 5.0 6.0 7.0 1.1 2.1 2.2 3.1 3.2 3.3 1.1
\end{Soutput}
\begin{Sinput}
> boot$X <- NULL
\end{Sinput}
\end{Schunk}
It looks like we have 14 estimated parameters.  We will map them to the
original control stream.
\begin{Schunk}
\begin{Sinput}
> boot <- boot[!is.na(text2decimal(boot$parameter)),]
> head(boot)
\end{Sinput}
\begin{Soutput}
  tool run parameter   moment     value
4  nm7   1    THETA1 estimate   7.98893
5  nm7   1    THETA1     prse      <NA>
6  nm7   1    THETA2 estimate    19.892
7  nm7   1    THETA2     prse      <NA>
8  nm7   1    THETA3 estimate 0.0650249
9  nm7   1    THETA3     prse      <NA>
\end{Soutput}
\begin{Sinput}
> unique(boot$moment)
\end{Sinput}
\begin{Soutput}
[1] "estimate" "prse"    
\end{Soutput}
\begin{Sinput}
> unique(boot$value[boot$moment=='prse'])
\end{Sinput}
\begin{Soutput}
[1] NA
\end{Soutput}
\end{Schunk}
prse, and therefore moment, is noninformative for these bootstraps.
\begin{Schunk}
\begin{Sinput}
> boot <- boot[boot$moment=='estimate',]
> boot$moment <- NULL
> unique(boot$tool)
\end{Sinput}
\begin{Soutput}
[1] "nm7"
\end{Soutput}
\begin{Sinput}
> boot$tool <- NULL
> head(boot)
\end{Sinput}
\begin{Soutput}
   run parameter     value
4    1    THETA1   7.98893
6    1    THETA2    19.892
8    1    THETA3 0.0650249
10   1    THETA4   3.35627
12   1    THETA5   123.566
14   1    THETA6   1.18258
\end{Soutput}
\begin{Sinput}
> unique(boot$value[boot$parameter %in% c('OMEGA2.1','OMEGA3.1','OMEGA3.2')])
\end{Sinput}
\begin{Soutput}
[1] "0"
\end{Soutput}
\begin{Sinput}
> unique(boot$parameter[boot$value=='0'])
\end{Sinput}
\begin{Soutput}
[1] "OMEGA2.1" "OMEGA3.1" "OMEGA3.2"
\end{Soutput}
\end{Schunk}
Off-diagonals (and only off-diagonals) are noninformative.
\begin{Schunk}
\begin{Sinput}
> boot <- boot[!boot$value=='0',]
> any(is.na(as.numeric(boot$value)))
\end{Sinput}
\begin{Soutput}
[1] FALSE
\end{Soutput}
\begin{Sinput}
> boot$value <- as.numeric(boot$value)
> head(boot)
\end{Sinput}
\begin{Soutput}
   run parameter       value
4    1    THETA1   7.9889300
6    1    THETA2  19.8920000
8    1    THETA3   0.0650249
10   1    THETA4   3.3562700
12   1    THETA5 123.5660000
14   1    THETA6   1.1825800
\end{Soutput}
\end{Schunk}
\subsection{Restrict data to 95 percentiles.}
We did 300 runs.  Min and max are strongly dependent on number of runs, since 
with an unbounded distribution, (almost) any value is possible with enough sampling.
We clip to the 95 percentiles, to give distributions that are somewhat more
scale independent.
\begin{Schunk}
\begin{Sinput}
> boot <- inner(
+ 	boot, 
+ 	preserve='run',
+ 	id.var='parameter',
+ 	measure.var='value'
+ )
> head(boot)
\end{Sinput}
\begin{Soutput}
  run parameter       value
1   1    THETA1   7.9889300
2   1    THETA2  19.8920000
3   1    THETA3   0.0650249
4   1    THETA4   3.3562700
5   1    THETA5 123.5660000
6   1    THETA6   1.1825800
\end{Soutput}
\begin{Sinput}
> any(is.na(boot$value))
\end{Sinput}
\begin{Soutput}
[1] TRUE
\end{Soutput}
\begin{Sinput}
> boot <- boot[!is.na(boot$value),]
\end{Sinput}
\end{Schunk}
\subsection{Recover parameter metadata from a specially-marked control stream.}
We want meaningful names for our parameters.  Harvest these from a reviewed control
stream.
\begin{Schunk}
\begin{Sinput}
> stream <- readLines('../nonmem/ctl/1005.ctl')
> tail(stream)
\end{Sinput}
\begin{Soutput}
[1] ";<parameter name='SIGMA1.1' label='$\\sigma^{1.1}prop$'>proportional error</parameter>"
[2] ""                                                                                      
[3] "$ESTIMATION MAXEVAL=9999 PRINT=5 NOABORT METHOD=1 INTER MSFO=./1005.msf"               
[4] "$COV PRINT=E"                                                                          
[5] "$TABLE NOPRINT FILE=./1005.tab ONEHEADER ID AMT TIME EVID PRED IPRE CWRES"             
[6] "$TABLE NOPRINT FILE=./1005par.tab ONEHEADER ID TIME CL Q V2 V3 KA ETA1 ETA2 ETA3"      
\end{Soutput}
\begin{Sinput}
> doc <- ctl2xml(stream)
> doc
\end{Sinput}
\begin{Soutput}
 [1] "<document>"                                                                                                                                                 
 [2] "<parameter name='THETA1' latex='$\\theta_1$' unit='$L/h$'    label='CL/F' model='$CL/F \\sim \\theta_6^{MALE} * (WT/70)^{\\theta_7}$'>clearance</parameter>"
 [3] "<parameter name='THETA2' latex='$\\theta_2$' unit='$L$'      label='Vc/F' model='$Vc/F \\sim (WT/70)^{1}$'   >central volume</parameter>"                   
 [4] "<parameter name='THETA3' latex='$\\theta_3$' unit='$h^{-1}$' label='Ka'                                     >absorption constant</parameter>"               
 [5] "<parameter name='THETA4' latex='$\\theta_4$' unit='$L/h$'    label='Q/F'                                    >intercompartmental clearance</parameter>"      
 [6] "<parameter name='THETA5' latex='$\\theta_5$' unit='$L$'      label='Vp/F'                                   >peripheral volume</parameter>"                 
 [7] "<parameter name='THETA6' latex='$\\theta_6$'                 label='Male.CL'                                >male effect on clearance</parameter>"          
 [8] "<parameter name='THETA7' latex='$\\theta_7$'                 label='WT.CL'                                  >weight effect on clearance</parameter>"        
 [9] "<parameter name='OMEGA1.1' label='$\\Omega^{1.1}CL/F$'>interindividual variability on clearance</parameter>"                                                
[10] "<parameter name='OMEGA2.2' label='$\\Omega^{2.2}Vc/F$'>interindividual variability on central volume</parameter>"                                           
[11] "<parameter name='OMEGA3.3' label='$\\Omega^{3.3}Ka$'>interindividual variability on Ka</parameter>"                                                         
[12] "<parameter name='SIGMA1.1' label='$\\sigma^{1.1}prop$'>proportional error</parameter>"                                                                      
[13] "</document>"                                                                                                                                                
\end{Soutput}
\begin{Sinput}
> params <- unique(boot[,'parameter',drop=FALSE])
> params$defs <- lookup(params$parameter,within=doc)
> params$labels <- lookup(params$parameter,within=doc,as='label')
> params
\end{Sinput}
\begin{Soutput}
   parameter                                          defs              labels
1     THETA1                                     clearance                CL/F
2     THETA2                                central volume                Vc/F
3     THETA3                           absorption constant                  Ka
4     THETA4                  intercompartmental clearance                 Q/F
5     THETA5                             peripheral volume                Vp/F
6     THETA6                      male effect on clearance             Male.CL
7     THETA7                    weight effect on clearance               WT.CL
8   OMEGA1.1      interindividual variability on clearance $\\Omega^{1.1}CL/F$
9   OMEGA2.2 interindividual variability on central volume $\\Omega^{2.2}Vc/F$
10  OMEGA3.3             interindividual variability on Ka   $\\Omega^{3.3}Ka$
11  SIGMA1.1                            proportional error $\\sigma^{1.1}prop$
\end{Soutput}
\begin{Sinput}
> boot$parameter <- lookup(boot$parameter,within=doc,as='label')
> head(boot)
\end{Sinput}
\begin{Soutput}
  run parameter       value
1   1      CL/F   7.9889300
2   1      Vc/F  19.8920000
3   1        Ka   0.0650249
4   1       Q/F   3.3562700
5   1      Vp/F 123.5660000
6   1   Male.CL   1.1825800
\end{Soutput}
\end{Schunk}
\subsection{Create covariate plot.}
Now we make a covariate plot for clearance.  We will normalize clearance 
by its median (we also could have used the model estimate).  We need to take 
cuts of weight, since we can only really show categorically-constrained distributions.
Male effect is already categorical.  I.e, the reference individual has median
clearance, is female, and has median weight.
\subsubsection{Recover original covariates for guidance.}
\begin{Schunk}
\begin{Sinput}
> covariates <- read.csv('../data/derived/phase1.csv',na.strings='.')
> head(covariates)
\end{Sinput}
\begin{Soutput}
     C ID TIME SEQ EVID  AMT    DV SUBJ HOUR TAFD  TAD LDOS MDV HEIGHT WEIGHT
1    C  1 0.00   0    0   NA 0.000    1 0.00 0.00   NA   NA   0    174   74.2
2 <NA>  1 0.00   1    1 1000    NA    1 0.00 0.00 0.00 1000   1    174   74.2
3 <NA>  1 0.25   0    0   NA 0.363    1 0.25 0.25 0.25 1000   0    174   74.2
4 <NA>  1 0.50   0    0   NA 0.914    1 0.50 0.50 0.50 1000   0    174   74.2
5 <NA>  1 1.00   0    0   NA 1.120    1 1.00 1.00 1.00 1000   0    174   74.2
6 <NA>  1 2.00   0    0   NA 2.280    1 2.00 2.00 2.00 1000   0    174   74.2
  SEX  AGE DOSE FED SMK DS CRCN predose zerodv
1   0 29.1 1000   1   0  0 83.5       1      1
2   0 29.1 1000   1   0  0 83.5       0      0
3   0 29.1 1000   1   0  0 83.5       0      0
4   0 29.1 1000   1   0  0 83.5       0      0
5   0 29.1 1000   1   0  0 83.5       0      0
6   0 29.1 1000   1   0  0 83.5       0      0
\end{Soutput}
\begin{Sinput}
> with(covariates,constant(WEIGHT,within=ID))
\end{Sinput}
\begin{Soutput}
[1] TRUE
\end{Soutput}
\begin{Sinput}
> covariates <- unique(covariates[,c('ID','WEIGHT')])
> head(covariates)
\end{Sinput}
\begin{Soutput}
   ID WEIGHT
1   1   74.2
16  2   80.3
31  3   94.2
46  4   85.2
61  5   82.8
76  6   63.9
\end{Soutput}
\begin{Sinput}
> covariates$WT <- as.numeric(covariates$WEIGHT)
> wt <- median(covariates$WT)
> wt
\end{Sinput}
\begin{Soutput}
[1] 81
\end{Soutput}
\begin{Sinput}
> range(covariates$WT)
\end{Sinput}
\begin{Soutput}
[1]  61 117
\end{Soutput}
\end{Schunk}
\subsubsection{Reproduce the control stream submodel for selective cuts of a continuous covariate.}
In the model we normalized by 70 kg, so that cut will have null effect.
Let's try 65, 75, and 85 kg. We have to make a separate column for each
cut, which is a bit of work. Basically, we make two more copies of our
weight effect columns, and raise our normalized cuts to those powers, 
effectively reproducing the submodel from the control stream.
\begin{Schunk}
\begin{Sinput}
> head(boot) 
\end{Sinput}
\begin{Soutput}
  run parameter       value
1   1      CL/F   7.9889300
2   1      Vc/F  19.8920000
3   1        Ka   0.0650249
4   1       Q/F   3.3562700
5   1      Vp/F 123.5660000
6   1   Male.CL   1.1825800
\end{Soutput}
\begin{Sinput}
> unique(boot$parameter)
\end{Sinput}
\begin{Soutput}
 [1] "CL/F"                "Vc/F"                "Ka"                 
 [4] "Q/F"                 "Vp/F"                "Male.CL"            
 [7] "WT.CL"               "$\\Omega^{1.1}CL/F$" "$\\Omega^{2.2}Vc/F$"
[10] "$\\Omega^{3.3}Ka$"   "$\\sigma^{1.1}prop$"
\end{Soutput}
\begin{Sinput}
> clearance <- boot[boot$parameter %in% c('CL/F','WT.CL','Male.CL'),]
> head(clearance)
\end{Sinput}
\begin{Soutput}
   run parameter    value
1    1      CL/F 7.988930
6    1   Male.CL 1.182580
7    1     WT.CL 1.308790
12   2      CL/F 7.636730
17   2   Male.CL 0.956565
18   2     WT.CL 2.369810
\end{Soutput}
\begin{Sinput}
> frozen <- data.frame(cast(clearance,run~parameter),check.names=FALSE)
> head(frozen)
\end{Sinput}
\begin{Soutput}
  run    CL/F  Male.CL   WT.CL
1   1 7.98893 1.182580 1.30879
2   2 7.63673 0.956565 2.36981
3   3 9.15198 0.937231 1.88593
4   4 9.56138 1.028670 1.47186
5   5 8.36964 0.914796 1.97656
6   6 9.09701 1.079030 1.16319
\end{Soutput}
\begin{Sinput}
> frozen$WT.CL65 <- (65/70)**frozen$WT.CL
> frozen$WT.CL75 <- (75/70)**frozen$WT.CL
> frozen$WT.CL85 <- (85/70)**frozen$WT.CL
\end{Sinput}
\end{Schunk}
\subsubsection{Normalize key parameter}
\begin{Schunk}
\begin{Sinput}
> cl <- median(boot$value[boot$parameter=='CL/F'])
> cl
\end{Sinput}
\begin{Soutput}
[1] 8.56139
\end{Soutput}
\begin{Sinput}
> head(frozen)
\end{Sinput}
\begin{Soutput}
  run    CL/F  Male.CL   WT.CL   WT.CL65  WT.CL75  WT.CL85
1   1 7.98893 1.182580 1.30879 0.9075635 1.094499 1.289313
2   2 7.63673 0.956565 2.36981 0.8389352 1.177625 1.584253
3   3 9.15198 0.937231 1.88593 0.8695648 1.138960 1.442193
4   4 9.56138 1.028670 1.47186 0.8966618 1.106883 1.330787
5   5 8.36964 0.914796 1.97656 0.8637440 1.146104 1.467795
6   6 9.09701 1.079030 1.16319 0.9174092 1.083560 1.253376
\end{Soutput}
\begin{Sinput}
> frozen[['CL/F']] <- frozen[['CL/F']]/cl
> head(frozen)
\end{Sinput}
\begin{Soutput}
  run      CL/F  Male.CL   WT.CL   WT.CL65  WT.CL75  WT.CL85
1   1 0.9331347 1.182580 1.30879 0.9075635 1.094499 1.289313
2   2 0.8919965 0.956565 2.36981 0.8389352 1.177625 1.584253
3   3 1.0689830 0.937231 1.88593 0.8695648 1.138960 1.442193
4   4 1.1168023 1.028670 1.47186 0.8966618 1.106883 1.330787
5   5 0.9776029 0.914796 1.97656 0.8637440 1.146104 1.467795
6   6 1.0625623 1.079030 1.16319 0.9174092 1.083560 1.253376
\end{Soutput}
\begin{Sinput}
> frozen$WT.CL <- NULL
> molten <- melt(frozen,id.var='run',na.rm=TRUE)
> head(molten)
\end{Sinput}
\begin{Soutput}
  run variable     value
1   1     CL/F 0.9331347
2   2     CL/F 0.8919965
3   3     CL/F 1.0689830
4   4     CL/F 1.1168023
5   5     CL/F 0.9776029
6   6     CL/F 1.0625623
\end{Soutput}
\end{Schunk}
\subsubsection{Plot.}
Now we plot.  We reverse the variable factor to give us top-down layout
of strips.
\begin{Schunk}
\begin{Sinput}
> levels(molten$variable)
\end{Sinput}
\begin{Soutput}
[1] "CL/F"    "Male.CL" "WT.CL65" "WT.CL75" "WT.CL85"
\end{Soutput}
\begin{Sinput}
> molten$variable <- factor(molten$variable,levels=rev(levels(molten$variable)))
> print(stripplot(variable~value,molten,panel=panel.covplot))
\end{Sinput}
\end{Schunk}
\includegraphics{model-covplot}
\subsubsection{Summarize}
We see that clearance is estimated with good precision.  Ignoring outliers, there 
is not much effect on clearance of being male, relative to female.  Increasing 
weight is associated with increasing clearance.  There is a 79 percent probability
that an 85 kg person will have at least 25 percent greater clearance than a 70 kg
person.
\section{Parameter Table}
\begin{Schunk}
\begin{Sinput}
> library(Hmisc)
> tab <- partab(1005,'../nonmem',tool='nm7',as=c('label','latex','model','estimate','unit','prse','se'))
> tab$estimate <- as.character(signif(as.numeric(tab$estimate),3))
> tab$estimate <- ifelse(is.na(tab$unit),tab$estimate,paste(tab$estimate, tab$unit))
> tab$unit <- NULL
> tab$label <- ifelse(is.na(tab$latex),tab$label,paste(tab$label, ' (',tab$latex,')',sep=''))
> tab$latex <- NULL
> names(tab)[names(tab)=='label'] <- 'parameter'
> tab$root <- signif(sqrt(exp(as.numeric(tab$estimate))-1),3)
> tab$estimate <- ifelse(contains('Omega|sigma',tab$parameter),paste(tab$estimate,' (\\%CV=',tab$root*100,')',sep=''),tab$estimate)
> tab$root <- NULL
> #offdiag <- contains('2.1',tab$parameter)
> #tab$estimate[offdiag] <- text2decimal(tab$estimate[offdiag])
> #omegablock <- text2decimal(tab$estimate[contains('Omega..(1|2)',tab$parameter)])
> #cor <- signif(half(cov2cor(as.matrix(as.halfmatrix(omegablock))))[[2]],3)
> #tab$estimate[offdiag] <- paste(sep='',tab$estimate[offdiag],' (COR=',cor,')')
> tab$model[is.na(tab$model)] <- ''
> #boot <- rlog(1:300,project='../nonmem/1005.boot',tool='nm7',boot=TRUE)
> boot <- read.csv('../nonmem/1005.boot/log.csv',as.is=TRUE)
> boot <- boot[boot$moment=='estimate',]
> boot <- data.frame(cast(boot,...~moment))
> boot[] <- lapply(boot,as.character)
> boot <- boot[contains('THETA|OMEGA|SIGMA',boot$parameter),c('parameter','estimate')]
> boot$estimate <- as.numeric(boot$estimate)
> boot <- data.frame(cast(boot,parameter~.,value='estimate',fun=function(x)list(lo=as.character(signif(quantile(x,probs=0.05),3)),hi=as.character(signif(quantile(x,probs=0.95),3)))))
> boot$CI <- with(boot, paste(sep='','(',lo,',',hi,')'))
> names(boot)[names(boot)=='parameter'] <- 'name'
> tab <- stableMerge(tab,boot[,c('name','CI')])
> tab$name <- NULL
> tab$se <- NULL
\end{Sinput}
\end{Schunk}
% latex.default(tab, file = "", rowname = NULL, caption = "Parameter Estimates from Population Pharmacokinetic Model Run 1005",      caption.lot = "Model 1005 Parameters", label = "p1005", where = "ht",      table.env = FALSE) 
%
\begin{table}[ht]
 \caption[Model 1005 Parameters]{Parameter Estimates from Population Pharmacokinetic Model Run 1005\label{p1005}} 
 \begin{center}
 \begin{tabular}{lllll}\hline\hline
\multicolumn{1}{c}{parameter}&\multicolumn{1}{c}{model}&\multicolumn{1}{c}{estimate}&\multicolumn{1}{c}{prse}&\multicolumn{1}{c}{CI}\tabularnewline
\hline
CL/F ($\theta_1$)&$CL/F \sim \theta_6^{MALE} * (WT/70)^{\theta_7}$&8.58 $L/h$&9.51&(7.14,9.89)\tabularnewline
Vc/F ($\theta_2$)&$Vc/F \sim (WT/70)^{1}$&21.6 $L$&9.33&(18.5,25.4)\tabularnewline
Ka ($\theta_3$)&&0.0684 $h^{-1}$&8.04&(0.0586,0.0793)\tabularnewline
Q/F ($\theta_4$)&&3.78 $L/h$&13.5&(3.03,4.83)\tabularnewline
Vp/F ($\theta_5$)&&107 $L$&15.7&(85.7,148)\tabularnewline
Male.CL ($\theta_6$)&&0.999&13.7&(0.799,1.31)\tabularnewline
WT.CL ($\theta_7$)&&1.67&21.9&(1.03,2.34)\tabularnewline
$\Omega^{1.1}CL/F$&&0.196 (\%CV=46.5)&23.1&(0.115,0.26)\tabularnewline
$\Omega^{2.2}Vc/F$&&0.129 (\%CV=37.1)&30.4&(0.0623,0.181)\tabularnewline
$\Omega^{3.3}Ka$&&0.107 (\%CV=33.6)&25.2&(0.0638,0.157)\tabularnewline
$\sigma^{1.1}prop$&&0.0671 (\%CV=26.3)&11.4&(0.055,0.0796)\tabularnewline
\hline
\end{tabular}

\end{center}

\end{table}\end{document}
